\chapter{Premi\`ere annexe}
\label{annexe:A}


\section{D\'efinitions}
\label{annexe:A:def}

\begin{frdefinition}\textbf {(Fonction continue)} \\
Soient $f:X\subseteq\mathbb{R}^n\rightarrow\mathbb{R}^m$ et $x_0\in X$. La fonction f est continue en $x_0$ 
si et seulement si
 $$\lim_{x\rightarrow x_0} f(x)=f(x_0)$$
c'est-\`a-dire que si, pour tout $\epsilon \in \mathbb{R}$, $\epsilon>0$, il existe $\eta>0$ tel que 
$$\lVert x-x_0\rVert < \eta \text{ et } x \in X \Rightarrow \lVert f(x)-f(x_0)\rVert < \epsilon$$
\end{frdefinition}


\begin{frdefinition}\textbf {(Valeurs et vecteurs propres)} \\
Soit une matrice carr\'ee $M\in \mathbb{R}^{n\times n}$. Les valeurs propres de $A$ sont les racines 
de son polyn\^ome caract\'eristique
$$p_M(z)=\det(zI-M)$$
o\`u $I$ est la matrice identit\'e de dimension $n$. Si $\lambda$ est une valeur propre de $A$, les vecteurs
$x\in \mathbb{R}^n$ non nuls tels que $$ Mx=\lambda x$$
sont appel\'es vecteurs propres.
\end{frdefinition}


\begin{frdefinition}\textbf {(Matrice d\'efinie positive)} \\
Une matrice carr\'ee $A \in \mathbb{R}^{n\times n}$ est dite d\'efinie lorsque \\
$$x^TAx>0, \ \forall x\in \mathbb{R}^n, \ x\neq 0$$
si de plus $A$ est sym\'etrique, toutes ses valeurs propres sont strictement positives. 
\end{frdefinition}



\begin{frdefinition}\textbf {(Matrice bande)} \\
Une matrice carr\'ee $A \in \mathbb{R}^{n\times n}$ est dite matrice bande si tout ses \'el\'ements 
en dehors de la bande diagonale born\'ee par deux constantes $k_1$ et $k_2$ sont nuls:
\begin{equation*}
 a_{ij}=0 \text{  si  }  j < i-k_1 \text{  ou  } j >i+k_2, \ k_1,k_2 \geq 0 
\end{equation*}



\end{frdefinition}





\begin{frdefinition}\textbf {(Fonction lipschitzienne)} \\
Soit $I$ un intervalle de $\mathbb{R}$ non vide et non r\'eduit \`a un point et 
$f:I\rightarrow \mathbb{R}$ une application alors on dit que f est 
$k$-lipschitzienne s'il existe un $k$ r\'eel strictement positif tel que
\begin{equation}
 \forall (x,y)\in I^2, |f(x)-f(y)|\leq k|x-y|
\end{equation}
La constante k est dite constante de Lipschitz.
\end{frdefinition}


