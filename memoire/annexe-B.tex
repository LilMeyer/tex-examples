


\chapter{Les difficult\'es rencontr\'ees}
\label{annexe:C}

\section{Librairie Mor\'e, Garbow et Hillstrom}


Sur le site de netlib\footnote{\url{http://www.netlib.org}}, on peut retrouver plusieurs librairies \'ecrites en fortran.
Celle qui nous int\'eresse se trouve plus pr\'ecis\'ement dans les probl\`emes sans contraintes\footnote{\url{http://www.netlib.org/uncon/data/}}.


Pour faciliter l'utilisation de tapenade, les fichiers sources ont dû être modifi\'es.
Tout d'abord, j'ai nomm\'e les fonctions par leur nom (\`a la place de {\tt getfun}) pour pouvoir avoir un appel
unique \`a la routine. J'ai remplac\'e tous les appels \`a la fonction {\tt ddot} par les op\'erations 
qu'effectue cette routine (produit scalaire). 
Il arrive qu'au lieu de passer un vecteur en argument, on donne un scalaire : {\tt g(j) = ddot( m, fj( 1, j), 1, f, 1)} l.155
 dans rose.f par exemple, ce qui a pour effet de multiplier la $j$\`eme colonne de {\tt fj} par {\tt f}. Cependant, Tapenade ne peut analyser
ce code pour le diff\'erentier parce qu'il s'agit d'une astuce sur l'incr\'ementation de {\tt fj(1,j)} qui donne {\tt fj(2,j)} ce qui est 
propre au fonctionnement interne de Fortran.




De plus, certains param\`etres ne sont initialis\'es que pour le mode $-1$ lors de l'appel \`a la fonction or 
ces param\`etres sont utilis\'es pour les autres modes. J'ai donc rajout\'e cette initialisation au d\'ebut de 
la fonction. Par exemple {\tt nqrtr  = 1} au d\'ebut sing.f.





\begin{itemize}
\item Changement du nom des routines getfun par le nom de la fonction (celui du fichier)
\item Remplacement des ddot par la somme des carr\'es
\item Remplacement des dcopy par copie \'el\'ements par \'el\'ements
\item Rajout des initialisations des param\`etres qui le sont uniquement dans le mode $-1$
\item parameter (one = 1.d0) n'est pas d\'efini dans badscp.f !! 
\item Rajout de {\tt ftf=0d0} pour \'eviter d'additionner les valeurs lorsque l'on fait plusieurs appels
\end{itemize}

L'ensemble de ces modifications peut être retrouv\'e dans le script {\tt make}.





\section{M\'ethodes d'ordres sup\'erieurs}

Comme le montre le tableau suivant, le point initial a une grande importance dans l'ex\'ecution des m\'ethodes. 
En choisissant un point al\'eatoire, il y a moins de chance pour que les algorithmes convergent vers le même point. D'autre part,
la convergence n'est pas garantie.

\begin{table}%[h]
	\begin{center}
{\small
%%%%%%%%%%%%%%%%%%%%%%%%%%%%%%%%%%%%%% Si la m\'ethode de Chebychev n'est pas une direction de Descente 
%on la garde quand meme!!!!!!!!!!!!!!!!!!!!!!!!!!!!!!!!!!!!!!!!!!!
\begin{tabular}{|l|c|c|c|c|c|c|}
  \hline fonction & $n$ & Iter Newton & Iter Cheb & temps Newton & temps Cheby & Norme diff \\
 \hline
   rose& 2 & 77 & 5 & 0.075 & 0.004 & 0.000000 \\\hline
   froth& 2 & 17 & 12 & 0.015 & 0.011 & 0.000000 \\\hline
   badscp& 2 & $19^*$ & 999 & 0.104 & 5.140 & 73081.313391 \\\hline
   badscb& 2 & 18 & 999 & 0.024 & 5.814 & 999915.027828 \\\hline
   beale& 2 & $14^*$ & 999 & 0.074 & 4.536 & 28470.397921 \\\hline
   jensam& 2 & 1 & 1 & 0.001 & 0.002 & Nan \\\hline
   helix& 3 & 14 & 10 & 0.013 & 0.014 & 0.000000 \\\hline
   bard& 3 & 9 & 2 & 0.009 & 0.002 & Nan \\\hline
   gauss& 3 & 1 & 1 & 0.001 & 0.001 & 0.000000 \\\hline
   meyer& 3 & $271^*$ & 999 & 0.302 & 6.159 & 6168.587560 \\\hline
   gulf& 3 & 1 & 1 & 0.001 & 0.001 & 0.000000 \\\hline
   box& 3 & 21 & 6 & 0.021 & 0.016 & Nan \\\hline
   sing& 4 & 26 & 19 & 0.024 & 0.017 & 0.000136 \\\hline
   wood& 4 & 27 & 21 & 0.025 & 0.024 & 0.000000 \\\hline
   kowosb& 4 & 230 & 2 & 0.221 & 0.003 & Nan \\\hline
   bd& 4 & 13 & 12 & 0.013 & 0.011 & 0.000000 \\\hline
   rosex& 10 & 203 & 5 & 0.207 & 0.005 & 0.000000 \\\hline
   singx& 12 & 30 & 24 & 0.030 & 0.023 & 0.000553 \\\hline
   pen1& 4 & 42 & 30 & 0.038 & 0.027 & 0.000000 \\\hline
   vardim& 10 & 26 & 19 & 0.025 & 0.018 & 0.000000 \\\hline
   trig& 10 & 20 & 2 & 0.023 & 0.027 & Nan \\\hline
   almost& 10 & 25 & 28 & 0.078 & 0.297 & Nan \\\hline
   bv& 10 & 22 & 999 & 0.022 & 5.438 & 185.331743 \\\hline
   ie& 10 & 21 & 17 & 0.025 & 0.029 & 0.000000 \\\hline
   band& 10 & 31 & 25 & 0.033 & 0.034 & 0.000000 \\\hline
   cheb& 10 & 167 & 999 & 0.208 & 5.010 & 1.674059 \\\hline
   ie & 100 & 89 & 360 & 2.251 & 11.208 & 0.000000 \\\hline 
   ie& 200 & 177 & 234 & 27.493 & 39.950 & 0.000000 \\\hline
   trig& 100 & 46 & 45 & 0.340 & 0.343 & 0.069866 \\\hline 
   trig& 250 & 61 & 58 & 2.104 & 2.114 & 14217.789748 \\\hline  
   trig& 350 & 61 & 65 & 4.270 & 4.700 & 0.022663 \\\hline  
   bv& 500 & 20 & 16 & 1.172 & 1.016 & 0.0066708 \\\hline
   bv& 1 & 19 & 15 & 5.404 & 4.585 &  0.1489379 \\\hline
   bv& 1500 & 17 & 14 & 10.86 & 9.613 & 12.905217 \\\hline
   bv& 2 & 17 & 13 & 19.097 & 15.863 & 17.806732 \\\hline
\end{tabular}
}
	\end{center}
	\caption{Nombre d'it\'eration des m\'ethodes de Newton et Chebychev mais sur un point initial loin de la 
solution. Les algorithmes convergent rarement au même point.}
	\label{tab:111}
\end{table}
